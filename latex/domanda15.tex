\section{Enunciare la legge di Bi\^ot-Sav\`art ed usarla per il
	calcolo del campo magnetico prodotto da una spira circolare
	percorsa da corrente stazionaria sull'asse della spira.}
Secondo la \textbf{Legge di Bi\^ot-Sav\`art} vale che :
$$
    d\vec{B} = \frac{\mu_0}{4 \pi} i \frac{d\vec{l} \times \hat{r}}{r^2}
$$
Dove : 
\begin{itemize}
	\item [$i$] {
	    Intensit\'a di corrente che scorre nella \textbf{spira circolare}.	
	}
	\item[$d\vec{l}$] {
	    Tratto di lunghezza \textbf{infinitesima} del filo della spira.
	}
	\item[$\vec{r}$] {
	    Raggio-vettore che individua il punto $P$ nel quale si vuole misurare il campo $d\vec{B}$.	
	}
\end{itemize}
INCOMPLETA!!!!
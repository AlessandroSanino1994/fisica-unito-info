\section{La legge di azione e reazione (terza legge di Newton)
    	per spiegare il moto di corpi a contatto. Enunciare la legge e
    	fornirne un esempio quantitativo.}
    La terza legge di Newton e' enunciata come :
    \begin{equation}
        \vec{F}_{A \rightarrow B} = - \vec{F}_{B \rightarrow A}
    \end{equation}
    \textbf{Ovvero:} \\
    "Se un corpo A esercita	una	forza $\vec{F}_{A \rightarrow B}$ su un corpo B, esso reagisce esercitando una forza $\vec{F}_{B \rightarrow A}$ sul corpo A. Le due forze hanno modulo e direzione uguali, ma verso opposto.”\\
	\\
	Prendiamo come esempio due corpi a contatto, ad esempio una persona e il pianeta Terra, da cui si puo' evincere che
	\begin{equation}
	    m_{persona} << m_{Terra} 
	\end{equation}
	\begin{equation}
	    \vec{F}_{Terra \rightarrow persona} = m_{persona} \vec{G}
	\end{equation}
	\begin{equation}
	    \vec{F}_{persona \rightarrow Terra} = m_{Terra} \vec{a} 
	\end{equation}
    $$ G \approx 9.81 \frac{m}{s^2} $$
	Le formule (3) e (4) sono una immediata conseguenza della seconda legge di Newton.\\
	sapendo che (1) \`e valida possiamo affermare che:

	$$ \vec{F}_{persona \rightarrow Terra} = -\vec{F}_{Terra \rightarrow persona} $$

	$$ \vec{a} \cdot m_{Terra} = -\vec{G} \cdot m_{persona} $$
	e siccome \`e valida (2) abbiamo che $a << G$. \\
	Questo \`e il motivo per cui sembra che sul pianeta Terra non sia esercitata alcuna forza dalla persona: L'accelerazione dovuta a essa \`e talmente piccola da risultare trascurabile.\\
	
	$\hfill\square$ % Q.E.D.
\section{Enunciare la legge di Faraday-Lenz e discutere la
	relazione con l'autoinduttanza.}
Secondo la \textbf{Legge di Faraday-Lenz} :
\begin{equation}
	\varepsilon_{indotta} = -\frac{d\Phi_S(\vec{B})}{dt}
\end{equation}
Ovvero a una \textbf{variazione del flusso} del campo magnetico corrisponde una forza elettromotrice autoindotta che si oppone a tale variazione. (il flusso del campo magnetico puo' variare, ad esempio, per una \textbf{variazione della intensita di corrente} $i$).\\
Come esempio di applicazione prendiamo un solenoide di lunghezza $l$:\\
Facciamo variare $i$ e ci accorgiamo della $\varepsilon$ autoindotta, introducendo di conseguenza una $i_{indotta}$ , di verso opposto a quello della corrente nel solenoide (fenomeno conosciuto come $autoinduzione$).\\
Sappiamo che:
$$
    \Phi_S(\vec{B}) = Li(t) = \oint_S{\vec{B} \cdot \hat{n}}
$$
Dove:
\begin{itemize}
	\item [$S$] {
	    \`E la superficie della \textbf{sezione} della spira.	
	}
	\item [$\vec{B}$] {
	    \`E il \textbf{campo magnetico}.
	}
	\item [$L$] {
		\`E il \textbf{Coefficiente di Induttanza} (si misura in Henry $H$)
	}
	\item[$i(t)$] {
	    \`E l'\textbf{intensit\'a di corrente} nel tempo.
	}
\end{itemize}
Pertanto (assumendo che la induttanza $L$ non vari nel tempo):
$$
    \Phi_S(\vec{B}) = Li(t) \rightarrow \frac{d\Phi_S(\vec{B})}{dt} = L\frac{di}{dt}
$$
$$
    \varepsilon_{indotta} = -L\frac{di}{dt}
$$

\pagebreak

\noindent Sfruttando la \textbf{legge di Ampere}, siccome siamo in un solenoide :
$$
    i_{conc} = n l i_{sol}
$$
\noindent Dove : 
\begin{itemize}
    \item [$i_{conc}$] {
    	Corrente concatenata alla circuitazione del solenoide.
    }
    \item [$n$] {
        Numero di spire (avvolgimenti) \textit{per unit\'a di lunghezza}.
    }
    \item [$l$] {
        Lunghezza del solenoide.
    }
    \item [$i_{sol}$] {
        Corrente che scorre nel solenoide.
    }
\end{itemize}
Pertanto: 
$$
    \varepsilon_{indotta} = -L\frac{di_{conc}}{dt} = -n l L \frac{di_{sol}}{dt}
$$
nel caso di un avvolgimento $n = 1$ quindi:
$$
    \varepsilon_{indotta} = -l L \frac{di_{sol}}{dt}
$$
$\hfill\square$
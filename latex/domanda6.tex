\section{Dimostrare che la forza di Coulomb è una forza
	conservativa e discutere brevemente le conseguenze di
	questo.}
Si dice che una forza \`e conservativa quando:
$$
   L_{\vec{r}_a \rightarrow \vec{r}_b} = U(B) - U(A)
$$
Si definisce il lavoro come :
$$
   L_{\vec{r}_a \rightarrow \vec{r}_b} = 
   \int_A^B{\vec{F_{\vec{r}_a \rightarrow \vec{r}_b}} \cdot \vec{ds}}
$$
Un'esempio di forza non conservativa \`e l'attrito.
Per quanto riguarda la forza di Coulomb invece si puo' affermare che:
\begin{displaymath}
    L_{\vec{r}_a \rightarrow \vec{r}_b} =
    \int_A^B{\vec{F_{\vec{r}_a \rightarrow \vec{r}_b}} \cdot \vec{ds}} =
    \int_A^B{ 
    	\frac{ 1 } { 4 \pi \varepsilon_0 } 
    	\frac{ q_1 q_2 }{ (r_b - r_a)^2 } 
        \hat{u}_{r_a \rightarrow r_b} \cdot \vec{ ds }
        }
\end{displaymath}
Definiamo $\vec{r} = \vec{r}_b - \vec{r}_a$ e riscriviamo come
\begin{displaymath}
    \int_A^B{ 
    	\frac{ 1 } { 4 \pi \varepsilon_0 } 
    	\frac{ q_1 q_2 }{ r^2 } 
    	\hat{u}_r \cdot \vec{ ds }
    } = 
    \int_A^B{ 
    	\frac{ 1 } { 4 \pi \varepsilon_0 } 
    	\frac{ q_1 q_2 }{ r^2 } 
    	\vec{dr}
    } =  
  	\frac{ 1 } { 4 \pi \varepsilon_0 } 
    \int_A^B{ 
    	\frac{ q_1 q_2 }{ r^2 } 
    	\vec{dr}
    }
\end{displaymath}
\begin{displaymath}
\frac{ 1 }{ 4 \pi \varepsilon_0 }
\left[ 
-\frac{ q_1 q_2 }{r}
\right]_A^B = 
\frac{ 1 }{ 4 \pi \varepsilon_0 }
\left(
    \frac{ q_1 q_2 }{r_A} - \frac{ q_1 q_2 }{r_B}
\right) = 
U(r_A) - U(r_B)
\end{displaymath}
Come conseguenza di questo si puo' affermare che il \textbf{Lavoro} compiuto dalla forza di Coulomb su un percorso chiuso \`e nullo:
$$
    L_{A \rightarrow B} + L_{B \rightarrow A} = 0
$$
Si puo' inoltre affermare che il Lavoro compiuto dalla Forza di Coulomb non dipende dal percorso fra A e B.
$\hfill\square$